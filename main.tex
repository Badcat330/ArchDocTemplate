
\documentclass[a4paper,18pt]{article} % добавить leqno в [] для нумерации слева

\linespread{1.6}

%%% Работа с русским языком
\usepackage{cmap}					% поиск в PDF
\usepackage{mathtext} 				% русские буквы в фомулах
\usepackage[T2A]{fontenc}			% кодировка
\usepackage[utf8]{inputenc}			% кодировка исходного текста
\usepackage[english,russian]{babel}	% локализация и переносы

%%% Гипер ссылки
\usepackage{xcolor}
\usepackage{hyperref}
\definecolor{linkcolor}{HTML}{000000} % цвет ссылок
\definecolor{urlcolor}{HTML}{191970} % цвет гиперссылок
\hypersetup{pdfstartview=FitH, citecolor=black, linkcolor=linkcolor,urlcolor=urlcolor, colorlinks=true}

%%% Дополнительная работа с математикой
\usepackage{amsmath,amsfonts,amssymb,amsthm,mathtools} % AMS
\usepackage{icomma} % "Умная" запятая: $0,2$ --- число, $0, 2$ --- перечисление

%% Номера формул
%\mathtoolsset{showonlyrefs=true} % Показывать номера только у тех формул, на которые есть \eqref{} в тексте.

%% Шрифты
\usepackage{euscript}	 % Шрифт Евклид
\usepackage{mathrsfs} % Красивый матшрифт

\usepackage{setspace}
\onehalfspacing

%%Картинки
\usepackage{graphicx}
\graphicspath{{../pictures/}}

%% Перенос знаков в формулах (по Львовскому)
\newcommand*{\hm}[1]{#1\nobreak\discretionary{}
{\hbox{$\mathsurround=0pt #1$}}{}}

% обновленная команда для нумерации секций
	\renewcommand{\thesection}{\arabic{section}}
	\setcounter{section}{0}


\title{\textbf{\Huge Архитектурный документ} \\ {\Large Название проекта}}
\author{Автор- БПИ183}
\date{\today}

\begin{document}

\maketitle
\newpage

\section{\textbf{Раздел регистрации изменений}}

\begin{table}[h]
    \flushleft
    \begin{tabular}{|c|c|c|c|}
        \hline
        \textbf{Версия документа} & \textbf{Дата} & \textbf{Описнаие изменения} & \textbf{Автор} \\
        \hline
    \end{tabular}
\end{table}

\newpage
\section{\textbf{Введение}}

\subsection{\textbf{Название проекта}}

\subsection{\textbf{Рамки проекта (Scope)}}

\section{\textbf{Общее описание архитектуры, задействованные архитектурные представления}}

\section{\textbf{Архитектурные факторы (цели и ограничения)}}


\newpage
\section{\textbf{Технические описания архитектурных решений}}

\subsection{\textbf{Техническое описание №1}}

\subsubsection{\textbf{Проблема}}
\subsubsection{\textbf{Идея решения}}
\subsubsection{\textbf{Факторы}}
\subsubsection{\textbf{Решение}}
\subsubsection{\textbf{Мотивировка}}
\subsubsection{\textbf{Неразрешенные вопросы}}
\subsubsection{\textbf{Альтернативы}}


\newpage
\section{\textbf{Представления архитектуры}}
\subsection{\textbf{Представление прецедентов (сценариев использования)}}

\subsection{\textbf{Логическое представление архитектуры}}

\subsection{\textbf{Представление архитектуры процессов}}

\subsection{\textbf{Физическое представление архитектуры}}

\subsection{\textbf{Представление развертывания}}

\subsection{\textbf{Представление архитектуры данных}}

\subsection{\textbf{Представление архитектуры безопасности}}

\subsection{\textbf{Представление реализации}}

\subsection{\textbf{Представление разработки}}

\subsection{\textbf{Представление производительности}}

\subsection{\textbf{Нефункциональные аспекты}}
\subsubsection{\textbf{Объем данных и производительность системы}}
\subsubsection{\textbf{Гарантии качества работы системы}}

\subsection{\textbf{Другие представления}}


\newpage

\addcontentsline{toc}{section}{Глоссарий}
\section*{Глоссарий}
\begin{itemize}
    \item Система управления базой данных (СУБД) -- совокупность программных и лингвистических средств общего или специального назначения, обеспечивающих управление созданием и использованием баз данных.
\end{itemize}

\bibliographystyle{unsrt}                                   
\bibliography{references}

\end{document}